\chapter{Discussões e Conclusões}\label{chp:conclusao}
\vspace{-1.5cm}
\noindent\rule{\columnwidth}{1.2mm}
\vspace{0.1cm}

O estudo apresentado, avaliou o sistema de iluminação, de uma edificação já existente, o Centro de Tecnologia e urbanismo da UEL, através do programa brasileiro de etiquetagem, tomando como base o regulamento técnico da qualidade para edificações Comerciais, de Serviço e Públicas, propôs melhorias necessárias para o alcance do nível A, na ENCE parcial do sistema de iluminação. Primeiramente, analisou-se o nível energético da edificação, e posteriormente, com as melhorias apresentadas, o impacto gerado por estas, na eficiência energética do edifício.

A edificação estudada, mostrou-se, no que diz respeito ao sistema de iluminação, muito eficiente, com base nas potências instaladas, o que surpreendeu positivamente, tendo em vista que uma parcela do edifício é uma construção antiga. No entanto, a classificação da ENCE parcial, foi penalizada, uma vez que a edificação não atende em sua totalidade, ao pré-requisitos específicos do sistema de iluminação, previstos no RTQ-C, que a credenciaria a uma ENCE de nível C. Porém como a edificação, mostrou-se eficiente ao realizar o método da área do edifício, foi necessário fazer uma ponderação, e a edificação recebeu uma classificação de nível B. Com as alterações sugeridas, é possível reverter essa situação, e assim a edificação, será credenciada a receber o nível A, na ENCE parcial do sistema de iluminação.

Com relação a análise das potências instaladas, as alterações sugeridas, mostram que é possível, reduzir esse valor em cerca de 55\% do valor atual instalado, tornando a edificação, que já mostra um valor aceitável, mais eficiente, reduzindo muito, o custo com energia elétrica, o que faz com que o investimento a ser realizado para a implementação, seja compensado à longo prazo.

O aproveitamento da luz natural, na edificação é um ponto importante a ressaltar, a alternativa de implementação de um sistema de dimerização, mostra-se muito interessante, tendo em vista, que com ele, é possível ter um bom aproveitamento da luz natural, e seu investimento, ao longo prazo acaba sendo compensado, com a redução dos gastos com energia elétrica pela edificação. É importante frisar que, para atender a este critério do RTQ-C, é necessário realizar o método da simulação.

Por fim, vale a pena ressaltar que o método prescritivo foi estabelecido como um conjunto de regras gerais para identificar a eficiência do
edifício e aplica-se à grande maioria de tipologias construídas atualmente no país. No entanto, ela não abrange todas as soluções possíveis de existir
em um edifício, e muitos casos só poderão ser avaliados pela simulação. Esta, por sua vez, pode incluir soluções que promovem a eficiência
energética e que não foram incluídas no método prescritivo, como aproveitamento da luz natural.



\vspace{-0.5cm}

\section{Trabalhos Futuros}

Esse trabalho, considerou apenas o sistema de iluminação, para o Centro de Tecnologia e Urbanismo da UEL. Assim, poderia ser realizado, futuramente, um estudo sobre os outros dois sistemas a serem avaliados no programa de etiquetagem brasileiro, a envoltória, e o sistema de condicionamento de ar. Neste caso, além de determinar o nível energético parcial de cada sistema, ou da edificação toda, poderia ser apresentadas sugestões para elevar esse nível energético.

Um outro estudo interessante que poderia ser realizado futuramente, é utilizar o método da simulação, tendo em vista que neste trabalho foi utilizado o método prescritivo, para poder realizar uma comparação entre os dois métodos, na obtenção da ENCE. 

