\chapter{Introdução}\label{chp:intro}
\vspace{-1.5cm}
\noindent\rule{\columnwidth}{1.2mm}
\vspace{0.1cm}

.
\section{Objetivos}

\begin{comment}
O objetivo desse trabalho é avaliar a instalação de uma usina fotovoltaica conectada à rede na Universidade Estadual de Londrina (UEL), sendo ela localizada ao lado da Farmácia Escola, tomando como base o edital $N^o$ 0217/2017, emitido no dia 25/10/2017. 

Para contextualizar, será apresentado os benefícios desse sistema fotovoltaico, demonstrando seus respectivos impactos: sociais, econômicos e ambientais gerados. 
\end{comment}
  

\section{Objetivos Específicos}

\begin{itemize}
\item Apresentar um estudo de caso de projeto elétrico de um sistema de armazenamento de grãos.
\item Obter o menor custo financeiro do projeto, atendendo qualidade e respeitando todas as normativas. 
\item Análise detalhada de cada etapa do projeto.

\end{itemize}

\section{Estrutura do Trabalho}
\begin{comment}
 Este trabalho está dividido em 5 capítulos. O primeiro capítulo será referente à uma introdução e contextualização do tema abordado, bem como a motivação do trabalho e os objetivos a serem alcançados.

No segundo capítulo será referente à uma revisão bibliográfica abordando os métodos de etiquetagem em geral, e posteriormente, focando no sistema de iluminação, apresentando o conceito de algumas soluções atuais disponíveis no mercado, que possam ser interessantes para possíveis implementações, onde será analisado a viabilidade financeira.

No terceiro capítulo será mostrado o método utilizado para a tomada de dados, bem como as soluções viáveis a serem expostas para uma possível implementação.

No quarto capítulo será mostrado os dados obtidos através das medições para a obtenção da ENCE do sistema de iluminação, bem como, as projeções com as possíveis implementações das soluções adotadas no trabalho.

No quinto e último capítulo, será abordado as discussões e conclusões com base nos dados obtidos, e nas projeções para as soluções encontradas.

\end{comment}


